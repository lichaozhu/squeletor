\begin{center}
\section*{Digital ou Numérique : un phénomène d'emprunt au c\oe ur de la start-up nation ?} 
\addcontentsline{toc}{section}{\protect\numberline{} {numérique \newline 
\textbf{Lichao Zhu et Gaël Lejeune} }}
\indexer{Zhu}{Lichao}{lichao.zhu@gmail.com} \up{1,2} et
\indexer{Lejeune}{Gaël}{gael.lejeune@sorbonne-universite.fr} \up{2}

{\small \up{1} Textes Théories Numérique (TTN), Université Paris XIII}
{\small \up{2} Sens Texte Informatique Histoire (STIH), Sorbonne Université}

\end{center}


%Un autre point de vue sur le sujet : https://www.lemondemoderne.media/numerique-et-politique-monde-moderne/
%\section{Introduction}
 Bien que l'anglais ne soit pas devenu la \textit{Lingua Franca} de l'Internet,  il est indéniable que son influence est très grande pour de nombreux domaines pour lesquels la circulation des connaissances se fait principalement par voie électronique\cite{Crystal-2002}. 
  Ceci se manifeste de nombreuses manières dont, pour ce qui intéresse la linguistique, les phénomènes d'emprunt et de calque.
   Nous nous intéresserons dans cet article à des phénomènes d'emprunts, d'adoption par une langue d'éléments langagiers provenant d'une autre langue. 
  
 Parmi les domaines pour lesquels l'influence de l'anglais est prégnante figurent en particulier la communication et l'informatique.
 Cette influence est visible sur la terminologie, on voit par exemple que dans le domaine de l'informatique le terme \textit{implémenter}, emprunté à l'anglais \textit{implement} supplante dans le français oral comme dans le français écrit le terme existant \textit{implanter} pour désigner l'activité de mise en place d'un programme\footnote{Voir par exemple sur le site de l'académie française un bref article sur le sujet : \url{http://www.academie-francaise.fr/implementer}}.
 Il est intéressant de remarquer que certains usagers de la terminologie informatique jugent que l'on pourrait conserver les deux termes \textit{implanter} et \textit{implémenter} mais avec deux acceptions différentes\footnote{Voir par exemple : \url{http://jargonf.org/wiki/implémenter}}.
%  On voit aussi que dans le langage courant, les emprunts sont également nombreux par exemple, dans un domaine connexe, dans tout ce qui a trait aux jeux vidéos
  
% Pour ce qui est de la communication, on peut observer l'irruption de : conf call ...
  La paire emprunt/terme supplanté qui nous intéresse pour cette étude est également issue de la terminologie informatique mais elle a quitté le domaine purement terminologique pour intégrer la langue courante.
Il s'agit de la paire digital(e)/numérique, l'observation donc de l'utilisation de "digital", comme adjectif ou comme nom, en remplacement de "numérique".
 La raison de notre intérêt pour cette paire est triple :
 \begin{itemize}
     \item On se situe à l'intersection de deux domaines (informatique et communication) dans lesquels les phénomènes d'emprunt, et particulièrement d'anglicismes, sont particulièrement foisonnants;
     \item Le terme français "numérique" est répandu, facile à écrire et à prononcer et disposait de surcroît d'une certaine antériorité de sorte qu'il aurait pu être à l'abri de la supplantation par un emprunt;
     \item Il s'agit d'un néologisme sémantique \cite{Moeschler-1974, Sablayrolles-2012} puisque la forme "digitale" dans le sens "relatif aux doigts" est préexistante à son usage dans le sens de "numérique") ce qui n'est pas sans occasionner un certain nombre de réalisations langagières malheureuses (ou amusantes selon le point de vue où l'on se place).
 \end{itemize}

Ce dernier aspect est particulièrement intéressant à étudier en diachronie, voire par exemple des travaux récents\cite{Jurafski-2016, Cartier-2016}.
%\begin{figure}[h!]
%\centering
%\includegraphics[scale=1.7]{universe}
%\caption{The Universe}
%\label{fig:universe}
%\end{figure}

%\section{Origines}

 \textit{Digital} a pour première acception "Qui a la forme d'un doigt" et "Relatif au doigt; qui fait partie du doigt." et trouve son étymologie en le mot latin impérial \textit{digitalis} dont la signification est "qui a la grosseur d'un doigt".

Son autre acception est "Qui est exprimé par un nombre, qui utilise un système d'informations, de mesures à caractère numérique." et trouve son étymologie dans le langage informatique des années 1960 en anglais, en particulier dans l'unité lexicale "\textit{digital computer}".
 Le trésor de la langue française informatisé précise par ailleurs les relations entre ces deux significations : "digital notamment dans digital computer « ordinateur digital » (du subst. digit « doigt » mais aussi « chiffre, [primitivement « compté sur les doigts »] »). % Les combinaisons que 

Tandis que \textit{numérique} signifie "Qui concerne des nombres, qui se présente sous la forme de nombres ou de chiffres, ou qui concerne des opérations sur des nombres." et "Qui désigne ou représente des nombres ou des grandeurs physiques au moyen de chiffres".% L

 Nous avons étudié l'usage des deux éléments de cette paire dans le corpus du journal le Monde de 1987 à 2017.
 La première grande vague d'utilisation de digital pour amener la notion de nombre se situe dans les années 1980-1990 autour des expressions "son digital", "écran digital" et affichage digital. Ce qui n'est pas sans causer des incompréhensions pour les locuteurs puisque la plupart des occurrences de digital en tant qu'adjectif se retrouve dans "empreinte(s) digitale(s)".
 Si le son est parfois "numérique", l'affichage et l'écran ne le sont que très rarement. A partir de 2002 environ, l'usage de digital est presque systématique pour décrire ces réalités.
 En effet, pendant la même période, l'adjectif numérique se retrouve principalement associé à d'autres noms : photo, télévision et bouquet.
Là encore, la différence dans le corpus est assez nette : "photo digitale" est très rare de même que "télévision digitale" et "bouquet digital".
 
 Au-delà des objets, telles que les décrivent les expressions citées ci-dessus, les processus arrivent rapidement au cœur des préoccupations exprimées par les journalistes. Or si la transformation est plutôt digitale, la fracture, elle, est surtout numérique. Ce phénomène est encore plus prégnant si l'on sort du corpus du Monde et de son écriture plus académique que d'autres supports: l'expression "fracture digitale" amène 8.000 résultats sur le moteur de recherche Google contre 2.000.000 pour "fracture numérique". S'il faut bien sûr prendre des pincettes avec ce genre de tests, la différence d'ordre de grandeur est très significative. D'ailleurs, si dans les articles sur la "transformation", les auteurs prennent la peine de citer les deux adjectifs, c'est rarement vrai pour ceux concernant la "fracture".
%dispositif
On observe également que l'emprunt est nettement moins fréquent au pluriel, ce qui est peut être dû à des problèmes d'adaptation morphologique avec des noms masculins \cite{Anas-2011}.
Au niveau de l'usage en tant que nom, il est intéressant de noter que le processus de "bas niveau" consistant à convertir des données dans un format traitable par un ordinateur reçoit le terme de "numérisation" beaucoup plus que "digitalisation".
%\section{Quelques éléments statistiques}

Nous présenterons un panorama plus large de ces usages en corpus, en comparant notamment les usages dans différents types de presse en ligne, dans le discours institutionnel (au sens politique) et les usages dans les forums et les documents de type "Présentation Power Point". Loin d'une vocation prescriptive, notre contribution viserait à présenter cette dualité dans une perspective tenant compte des publics visés et des types de discours (voire par exemple \cite{Salvador-2017}.

\bibliographystyle{plain}
\bibliography{references}
